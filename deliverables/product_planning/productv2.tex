\subsection{High-level product backlog}
	This section describes the desired features of the final application according to the MoSCoW method \cite{brennan2009guide}. These features are categorized into four groups:
	
	\begin{itemize}
		\item \textbf{Must have} 
			\subitem Requirements that must be statisfied in the final version
		\item \textbf{Should haves} 
			\subitem Requirements of high priority that should be included if possible
		\item \textbf{Could haves} 
			\subitem Requirements that are considered desirable but not necessary
		\item \textbf{Would haves} 
			\subitem Requirements that stakeholders have agreed will not be implemented in the final version, but could be added in the future
	\end{itemize}
	
	\subsubsection{Must haves}
		\begin{itemize}
			\item Visualize the triodata of the father, mother and child
			\item Visualize mutations and gene interaction
			\item Looking for known disease mutations
			\item Retrieving data from existing genetic databases
			\item Reading VCF-files
			\item Easy to use GUI for doctors, e.g. no redundancy
		\end{itemize}
	
	\subsubsection{Should haves}
		\begin{itemize}
			\item Save patient data in a database
			\item Secure login
			\item Uploading VCF files to the server in the background
		\end{itemize}
	
	\subsubsection{Could haves}
		\begin{itemize}
			\item Exporting visualization data
		\end{itemize}
	
	\subsubsection{Would haves}
		\begin{itemize}
			\item Spread computational power over multiple threads, cores, or systems
			\item Support for mobile web browsers
		\end{itemize}

\subsection{Roadmap}
	This is our major release schema. Because we are using scrum with sprints of one week, the end of every week indicates the end of a sprint. This also means that we will have a working product after each sprint, with new, more or improved features every week.

	\subsubsection*{Sprint 1 10/05-16/05}
		Basic setup of framework, read VCF files and connect to database.
	
	\subsubsection*{Sprint 2 17/05-23/05}
		Make useful query's for the database and find all mutations in the VCF file. Also set up the front end.
	
	\subsubsection*{Sprint 3 24/05-30/05}
		Retrieve data from the database associated with the mutations found in the VCF file and make a visualization of the mutations.
	
	\subsubsection*{Sprint 4 31/05-06/06}
		Visualize the found mutations and possible connections with genes associated with a disease.
	
	\subsubsection*{Sprint 5 07/06-13/06}
		Improve usability of the product.
	
	\subsubsection*{Sprint 6 14/06-20/06}
		Improve visualizations.
	
	\subsubsection*{Sprint 7 21/06-27/06}
		Finish report and presentation.