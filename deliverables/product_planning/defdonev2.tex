This section describes when a feature is really done and ready to be integrated in the system. We will define this for a feature, sprint and the end product.
\\\\
A feature is finished when it is fully tested and the code is accepted by other developers.
The tests should be implemented with JUnit and the code coverage should be high enough.
Furthermore, the code must be fully documented with JavaDoc and should match the rules of CheckStyle.
\\\\
A sprint is finished when the whole application is tested and approved, just like a feature. However the continuous integration system should also accept the build. Next, the developers and users will test the system by hand to check for bugs.
\\\\
The end product is finished if all the Must- and Should haves of the MoSCoW model are implemented and tested as described above. The reason we are not content with just the Must haves is because only the Must haves result in a very basic application. The product should be approved by the stakeholders based on the looks and feels, and they should be happy with the product.
\\\\
Furthermore in addition to all this, the code should be well documented, tested, style checked and integrated. This will all be evaluated by the SIG (Software Improvement Group) and should be improved after the first check.