GEVATT is developed in quite a short time. The course Contextproject lasts about ten weeks and in this ten weeks the team had to become skilled in the context, make accompanying documents	 and present the final product. So only a limited part of the time is actually used to develop the application. This means there's focussed on the main parts of the application and there are more functionalities that could be added later if development would continue. This chapter is about what functionalities  could be added.

One thing is exporting; this is currently not possible in the application. The found mutations and information from the VCF files are only visible in the application itself and require logging in. For a doctor, it might be handy to export some data and visualisations to make it easier to present them. E.g. it's easier for a doctor to show a print of visualisations to his patients than showing a screenshot of the application. It also might be handy to have the data exported to a spreadsheet format.

The application is web based and therefor in principal platform independent: each device with a modern browser and an internet connection could open the application. Something that is limited to the use of the application now is the minimum screen width. When a device with a resolution of lower than 1024px is used, the user need to scroll to see all information. The application could be modified so that the layout adopts to smaller screens and become thereby usable on tablets and phones. A nice feature, but is questionable if a doctor would use the application on such devices.

Currently the application is mainly focused on individual mutations found in the uploaded data, and less on the relations between these mutations. When development of this application would continue, it will probably make sense to focus on implementing/extending visualisations that show these relations.

Testing is only done on a low scale. If GEVATT would be used in production, there should be some testing be done based on heavy use. Furthermore, if the application would be used in producteion, some managing features should be added. For example, creating accounts en revoking access to specific doctors. Due to time considerations this is left out in this project so far.