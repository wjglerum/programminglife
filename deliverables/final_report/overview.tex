The developed product, a secured web application, is built on de Play Framework. It contains several parts that work together to deliver a user friendly environment to explore genetic information.

\subsection{Authentication}

Starting with the basis, it is a secured web application where a user needs to log in. The application is secured because sensitive information is used and stored (namely genetic information) and patients probably do not want this information to be public. It is a web application so it is easily accessible from multiple devices and updates do not have to be done locally. After logging in some secured pages become accessible. Some other pages are already accessible before logging in, e.g. the 'documentation' and 'about' pages. We distinguish the following parts:

\begin{itemize}
  \item A login page
  \item Secure pages that are not publicly accessible
  \item Redirect if secured page is requested while unauthenticated
  \item Prevent doctors from accessing data of patients of other doctors
\end{itemize}

\subsection{Context specific}

The web framework Play and it is built-in secure module is not developed by ourselves so we could focus on developing the context specific parts. This was mainly separable in three parts: analysing data, retrieval of relevant data from databases and visualising data.

\subsubsection{Data analysis}

Analysing the data is done by processing the VCF file and detecting mutations. The outcome of the processing is used in the visualisation. At both of these parts there is information used given by some databases. The information is used to get more information about mutations and find relations between several mutations. The application does the following:

\begin{itemize}
  \item Read the VCF file and find mutations of two types: de novo's and recessive homozygous
  \item Save metadata about the uploaded file, like the filename, file size, etc.
  \item Find relations between found mutations
\end{itemize}

\subsubsection{Data retrieval}

Information from multiple databases is used for both visualisation and finding relations between mutations. This includes:

\begin{itemize}
  \item Manage connections to multiple databases
  \item Querying databases to get relevant information from mutations and proteins
\end{itemize}

\subsubsection{Visualisation}

The most important part of the application: visualisation. Multiple views of the same found mutations are made to give as much insight as possible. This includes the following visualisations:

\begin{itemize}
  \item An overview of all mutations in a patients VCF file with a view per chromosome
  \item A distinct page per mutation, showing:
  \begin{itemize}
     \item The position of the mutation relative to nearby genes
     \item A graph with a protein related to the mutation, and proteins related to that protein
  \end{itemize}
\end{itemize}

These visualisations each help to meet the requirements because they each have a different view on the found mutations. The chromosome overview gives a direct overview of everything found. The other two visualisations focus on two specific parts. The location of mutations relative to genes could help the doctor understand that a mutation is related to another gene and therefore probably a disease. The protein graph visualisation makes relations between other proteins related to mutations understandable. This could help a docter to identify symptoms or diseases caused by other proteins that are indirectly related to the mutation.