The application developed is called GEVATT, which stands for GEnetic Variations Analyzer Through Triodata, which is an application used by doctors to browse genetic information of patients. Such applications are also called genome browsers. It is a secured web application based on the Play Framework. Doctors need to login from any device with a browser in order to use the application.

The application makes it possible for a doctor to upload a VCF file and let the application analyse it. A VCF (Variant Call Format) file is a file used in bioinformatics for storing gene sequence variations. This storing of genetic information of patients is based on variations between de patient and a reference genome. This is done because storing al information would be taking too much space. The 

After uploading the file the user of the application, a doctor, waits till the file is processed. In the meanwhile the doctor could browse other patients he uploaded information about earlier. After uploading, the first important part of the applications is executed: analysing the data. The outcome of the analysation consists of mutations found in the genome of the patient and the relations between these mutations.

Secondly, the main focus of the application is about visualising the found mutations. This is firstly done by giving a main overview of the whole patient. An overview of all chromosomes is given and per chromosome is indicated if and how many mutations it contains. Besides a visual overview of the chromosomes there's a tabular overview given with all mutations and some extra information per mutation. This makes it easier to estimate which mutations are more harmful than others.

Most information is shown on the overview pages per mutation. The pages with these visualisations have two visualisations which both give another insight in the mutation. The first part shows the position of the mutation relatively to a gene. The second visualisation is a graph based interactive visualisation showing proteins related to the mutation and the connections between these proteins. Here we also have a tabular overview per protein with per protein information about which deceases they could cause and to which other mutations of the patient they are related.