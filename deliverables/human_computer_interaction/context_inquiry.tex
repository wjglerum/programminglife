\section{Context Inquiry}
A good idea is to plan a contextual inquiry, this means discovering the needs of the customer at the workplace.
First a small interview is held with the customer to get to know each other, acquire permission and explain the process and focus of the inquiry.
This will be followed by a inquiry of a couple of hours at the workplace.

There are a lot of genetic analyses tools available, so there are many people available for an inquiry.
We could discover more features by for example asking what the customer is missing. 
Then we ask the customer how it is done now and how it could be done easier.
Furthermore we could also improve existing features by asking what problems there occur while performing the task at hand.

For this project we did not have real customers to hold a context inquiry. 
However we could ask our teaching assistants to act as a customer and get to know their needs.